\section{Framework}\label{sec:framework}
\subsection{General description}
In Euclid code development and running has to be done in a specific framework, which is called EDEN (Euclid Development ENvironment). This environment is realized with a virtual machine, named LODEEN, which is maintained and versioned by the Euclid System Team. Essentially it is a CentOS machine, with also a Docker image version named DOCKEEN. More details at the end of this section.
The pre-defined working areain LODEEN is \verb+$HOME/WorkProjects+, where all the local version of the codes should be installed and maintained. Tagged official versions of Euclid codes are available at the path specified by the environment variable \verb+EUCLIDPROJECTPATH+.

\paragraph{Useful links}
\begin{itemize}
\item Developers Corner: \url{https://euclid.roe.ac.uk/projects/codeen-users/wiki/DevCorner}
\item DOCKEEN gitlab repository (ESAC credentials needed): \url{https://gitlab.euclid-sgs.uk/ST-TOOLS/CT_DOCKEEN/-/tree/2.1.2}
\end{itemize}

\paragraph{Useful environment variables}
\begin{itemize}
\item \verb+User_area+: it points to \verb+$HOME/Work/Projects+
\item \verb+EUCLIDPROJECTPATH+: it is the path where official tagged version of Euclid codes are installed
\end{itemize}


\subsection{The E-Run command}
The official tool for running Euclid codes is \verb+E-Run+, which is also aliased as \verb+ERun+ or \verb+EuclidRun+. This command runs a specific main of a given project with the syntax:

\begin{center}
\verb+E-Run <project> <project_version> <main> <parameters>+
\end{center}

Here the \verb+<parameters>+ argument should contain all the command line parameters which would be normally passed to the specified \verb+<main>+ program which is executed. In particular TIPS (whose project name is \verb+SIM_TIPS_Simulator+) is mainly written in Python, and its main programs have command line parameters are specified by the \verb+argparse+ module. Each parameter is defined by a keyword and it is assigned to a value, with the syntax \verb+--<keyword> <value>+. A practical example is

\begin{verbatim}
E-Run SIM_TIPS_Simulator 4.6.8 EuclidNisSplit
--workdir /path/to/workdir 
--inputNIS example_inputNIS.xml 
--mdb example_MDB.xml 
--starcat example_STARCAT.xml
--galcat example_GALCAT.xml  
--catalogsNIS example_TU.xml
\end{verbatim}

The \verb+<project_version>+ argument (here \verb+4.6.8+) is optional. If it is passed, the \verb+E-Run+ command will look for the version \verb+4.6.8+ of requested project, i.e. \verb+SIM_TIPS_Simulator+ in this example case. As explained above \verb+E-Run+ will look for the project starting from the path defined in the environment variable \verb+$EUCLIDPROJECTPATH+, which is mounted on the LODEEN environment on the \verb+cvmfs+ filesystem. On EDEN-2.0 this path is

\begin{verbatim}
/cvmfs/euclid-dev.in2p3.fr/CentOS7/EDEN-2.0/opt/euclid
\end{verbatim}